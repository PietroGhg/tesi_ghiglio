\documentclass[11pt]{article}
\usepackage[a4paper,margin=2cm]{geometry}
\usepackage{hyperref}
\usepackage[style=numeric]{biblatex}
\addbibresource{./ref.bib}
\usepackage{listings}
\lstset { %
    language=C++,
    %backgroundcolor=\color{black!5}, % set backgroundcolor
    basicstyle=\footnotesize,% basic font setting
}
\title{Rules for energy efficiency in C/C++}
\author{Pietro Ghiglio}
\begin{document}
\maketitle
\tableofcontents{}

\section{Introduction}
These are some rules that I've found in the papers that I have read in the previous months.
They mainly refer to the papers \cite{garcia} and \cite{moriso}.
For each rule I will provide its informal description and eventually a note when needed.

\section{Rules}
\begin{enumerate}
\item \textbf{Boolean return} \newline
Instead than returning the conjunction/disjunction of boolean variables, convert the multiple booleans into an unsigned int with associated flags and return its comparison with 0 or 1. \newline
Example:
\begin{lstlisting}
bool or4(bool a, bool b, bool c, bool d){
	return a || b || c || d;
}

#define flagA (1u << 0)
#define flagB (1u << 1)
#define flagC (1u << 2)
#define flagD (1u << 3)

bool or4opt(unsigned int mask){
	return (mask & (flagA | flagB | flagC | flagD)) != 0;
}

//usages
or4(1,0,0,0);
or4opt(1000);
\end{lstlisting}

\item \textbf{Row Major accessing} \newline


\item \textbf{Passing objects by reference} \newline
Passing an object or a struct by reference avoids the overhead of creating a complete copy of the argument. \newline
Example: \newline
\begin{lstlisting}
typedef struct {
	int a;
	int b;
} my_s;

int f(my_s value){
	return s.a+s.b;
}

int f_opt(my_s* ref){
	return s->a + s->b;
}
\end{lstlisting}
\end{enumerate}

\printbibliography[heading=bibintoc,title={References}]

\end{document}