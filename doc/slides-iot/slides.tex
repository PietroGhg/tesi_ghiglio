\documentclass{beamer}
\usepackage{textcomp} %textrightarrow
\usetheme{Boadilla}
\title{Deployment applicazioni IoT}
\author{Pietro Ghiglio}
\begin{document}
\titlepage

\begin{frame}
\frametitle{Osservazioni}
\begin{itemize}
\item Poche regole riguardo sistemi embedded/IoT.
\item In letteratura, molto di più riguardo al deployment.
\end{itemize}
\end{frame}

\begin{frame}
\frametitle{Paper 1}
Reference: Resource Aware Placement of IoT Application Modules in Fog-Cloud Computing Paradigm - Taneja.
\begin{itemize}
\item Approccio (molto) euristico.
\item Capacità di ogni nodo e requisiti di ogni modulo dell'applicazione.
\item CPU, RAM, Bandwidth.
\end{itemize}
\end{frame}

\begin{frame}
\frametitle{Paper 2}
Reference: A Discrete Particle Swarm Optimization Approach for Energy-Efficient IoT Services Placement Over Fog Infrastructures
- Djemai.
\begin{itemize}
\item Algoritmo di ottimizzazione più complesso.
\item Più parametri richiesti.
\item Nodi: CPU, RAM, potenza.
\item Link: tecnologia, bandwidth, latenza.
\item Moduli: tecnologia, requisiti di CPU, RAM, latenza.
\item Data dependencies: quantità di dati scambiati, massimo delay.
\item Massimo delay dell'applicazione tra sensori e attuatori.
\end{itemize}
\end{frame}

\begin{frame}
\frametitle{Paper 3}
Reference: Spinal Code: Automatic Code Extraction for Near-User Computation in Fogs - Bongjun
\newline\newline
\begin{itemize}
\item Code annotations + analisi statica.
\item Partizione delle istruzioni del programma tra i nodi della rete. Minimizza latency.
\item Aggiunta del codice di comunicazione in caso di data dependency.
\item Informazioni reperite a runtime + ricompilazione.
\end{itemize}

\end{frame}


\end{document}