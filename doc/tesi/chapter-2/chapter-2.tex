\section{Motivations}
Solita pappardella + paper interview sull'energia

\section{Instruction Level Energy modeling}
Given a target's Instruction Set Architecture (ISA), an energy model is a model of the energy consumed by each instruction.
They have been introduced in 1996 by Tiwari et al. \cite{tiwari}. 
\subsection{Characterization of an ISA Energy model}
The main components of an energy model are:
\begin{itemize}
\item Instruction base cost ($B_{i}$, for each instruction $i$): the cost associated with the basic processing needed to execute an instruction. 
\item Effect of circuit state ($O_{i,j}$, for each pair of instruction $i$, $j$): the cost of the switching activity resulting from executing two consecutive instructions differing one from another.
\item Other inter-instruction effects ($E_{k}$, for each additional effect $k$): any other effect that can occur in real program, such as stalls or cache misses.
\end{itemize}
\par Given these components and a program $P$, the total energy consumed by it, $E_{p}$, is given by:
\begin{gather*}
E_{p} = \sum_{i} (B_{i} \times N_{i}) + \sum_{i,j} (O_{i,j} \times N_{i,j}) + \sum_{k} E_{k}
\end{gather*}
Where $N_{i}$ is the number of occurrences of instruction $i$, and $N_{i,j}$ is the number of times there has been a switch from instruction $i$ to instruction $j$.

\subsection{Why employing an ISA energy model}
The most common way to describe a processor's power consumption is through the average power consumption. \par
This single number may not provide enough information to characterize the energy consumed by a program running on the target processor: different programs may employ the functional units of the CPU in different ways, leading to different measurements at equal running time. \par
ISA Energy Models offer a more detailed view of the energy profile of the target architecture. They therefore allow to identify variations of consumed energy from one program to another, and may also guide decision of both humans (hardware/software design) and software (compilers or operating systems).

\subsection{Producing an ISA energy model}
Energy models can be produced through an experimental procedure. \newline
In order to obtain instruction base costs, a program consisting of a large loop of a repeated instruction is written. Then one can measure the average current drawn by the processor while executing the program, $\hat{i}$, and multiply it by the supply voltage $V_{cc}$, obtaining the base energy consumption. \newline
Instruction may also be grouped together, since instruction with similar functionality will have similar base cost.\par 
In order to obtain the circuit state effects, loop of pairs of instruction are required. The difference between the instruction's base costs and the average current measured provides the circuit state overhead. \par 
A similar approach can be employed to obtain the costs of other inter-instruction effects: writing large loops in which the examined effect occurs several times, measuring the average current and subtracting the costs that are already known (base costs and circuit state). \par 
The main disadvantage of this approach is that several different programs must be written: for an ISA with $n$ instructions, $\bigO(n)$ programs are required to produce base costs and $\bigO(n^{2})$ for circuit state effects. \newline
Estimation of other inter-instruction effects also gets more difficult as the complexity of the architecture increases. \par
On the other hand, this approach has the big advantage of not requiring a model of the circuit of the target processor, information that is often not disclosed by the manufacturing companies.

\section{Energy consumption estimation}
Paper eder. 
da leggere less is more\\
xmos energy model: energy modeling of multi-threaded architecture. context switching has an energy cost. extends tiwari's model by adding thread switching cost. xmprofile software suite: allows to generate load and monitor test executions. tests are generated automatically. generate tests only for instructions with no effects on control flow and no non-deterministic timing.
observes that number of operands has significant impact on power consumption. data width has an impact on power consumption (low impact). they choose to generalize inter-instruction overhead. 
for instructions that cannot be directly tested, they propose two solutions: 1) group instructions by #operands, and put the untestable instr in the appropriate group. 2) assign default cost to untestable instr. 
they obtain execution statistics by hardware simulation (this can be replaced by profiling).
grouped model performs worse than individual model (16percent error vs 7percent error). both provide a consistent underestimation.
da leggere: lee, nunez-yanez

rieger: misurazione diretta della corrente (campo magnetico indotto) + metodo preso da NLP. dicono loro stessi che non funziona. \\
rieger-survey: alcuni tool commericali, altra letteratura. molto android/java (ex. powertutor)
roth: piattaforma hardware per creare energy models + xml dell'energy model per intergrarlo nel loro compilatore, il quale fornisce una stima worst case. \\
brandolese: \\  come eder è llvm-based, differiscono per ir-assembly mapping. eder ha anche metodo statico-worst case. \\
pereira: source level view, statistical method to provide ranking, no energy estimation, just visualization. \\
jrapl. only intel, code segment granularity. check intel rapl 


